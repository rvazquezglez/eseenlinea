\documentclass[12pt]{article}
\usepackage{amsmath}
\usepackage{amssymb}
\usepackage{latexsym}
\usepackage[spanish]{babel}
\usepackage[pdftex]{graphicx} % PDFLaTeX
\DeclareGraphicsExtensions{.png,.pdf,.jpg}
\usepackage{epsf}
\usepackage{graphicx}
\usepackage{mathrsfs}


\newtheorem{teorema}{Teorema}[section]
\newtheorem{lema}[teorema]{Lema}
\newtheorem{prop}[teorema]{Proposici\'on}
\newtheorem{corolario}[teorema]{Corolario}
\newtheorem{definicion}[teorema]{Definici\'on}
\newtheorem{problema}{Problema}
\newtheorem{ejemplo}{Ejemplo}
\spanishdecimal{.}
\usepackage[latin1]{inputenc}

\begin{document}

\title{\'ALGEBRA LINEAL}
\author{\textbf{ESCUELA SUPERIOR DE ECONOMIA}}
\date{\large{M\'exico, D.F., 10 de marzo de 2010}}
\def\contentsname{Contenido}
\maketitle
\tableofcontents
\newpage

\section{Espacios vectoriales}
\subsection{Espacio y subespacio  vectorial}

\begin{definicion}
Sea $K$ un campo. Un espacio vectorial sobre $K,$ o tambi\'en
llamado un $K$- espacio vectorial, consta de lo siguiente:
\begin{enumerate}
\item[1.] Un conjunto $V,$ cuyos elementos se llaman vectores.
\item[2.] Una operaci\'on binaria en $V,$ llamada suma de vectores, denotada por $+$, y que cumple lo
siguiente:
\begin{enumerate}
\item[a)] Para todos $x, y \in V,$ se cumple que $x + y = y + x$ (conmutatividad).
\item[b)] Para todos $x,y$ y $z \in V,$ se cumple que $(x + y) + z =x + (y + z)$ (asociatividad).
\item[c)] Existe un elemento en $V$ llamado cero y denotado por  0 tal que $0 + x = x$, pata todo $x \in V$ (existencia del neutro aditivo).
\item[d)] Para todo $x \in V$ existe un elemento $-x$ tal que $x + (-x) = 0$ (existencia de elementos inversos).
\end{enumerate}
\item[3.] Una operaci\'on binaria en $V,$  llamada producto de vectores, denotada por $\cdot$, y que cumple lo
siguiente:
\begin{enumerate}
 \item[a)] Para todo $x \in V$, se tiene que $1 \, x = x$, con $1 \in K.$
 \item[b)] Para todo $x \in V$ y para todo $\lambda$ y $\mu \in k$, se tiene que $\lambda(\mu x) = (\lambda\mu) x$.
 \item[c)] El producto por escalar es distributivo, es decir,
 \begin{align*}
(\lambda + \mu) x &=\lambda x + \mu x, \\
\lambda(x + y) &= \lambda x + \lambda y,
\end{align*}
para todos $\lambda,\mu \in K$ y para todos $x, y \in
V$
\end{enumerate}
\end{enumerate}

\begin{definicion}
Al conjunto $V$ con la suma y el producto por escalar se le llama \textbf{ espacio vectorial sobre $K$}.
\end{definicion}

\begin{ejemplo}
La operaci\'on de suma y producto por escalar en $\Bbb{R}^{3}$ se formulan como:

\begin{enumerate}
  \item Dados $(x_1,x_2,x_3)$, $(y_1,y_2,y_3) \in \Bbb{R}^3$, se define:
  \[
     (x_1,x_2,x_3) + (y_1,y_2,y_3) = (x_1 + y_1,x_2 + y_2, x_3+y_3)
   \]
   \item Dados $(x_1,x_2,x_3) \in \Bbb{R}^3$ y $c \in \Bbb{R}$, se define:
   \[
    c\,\,  (x_1,x_2,x_3) = (cx_1,cx_2,cx_3)
   \]
\end{enumerate}

Entonces $\Bbb{R}^3$ con la suma y producto definidos anteriormente
es un espacio vectorial. Para esto verifiquemos que $\Bbb{R}^3$ con
la operaci\'on $+$ cumple las siguientes propiedades
\begin{enumerate}
\item[a)] Para todos $x, y \in \Bbb{R}^3$, se cumple que $x + y = y + x$ (conmutatividad).
 Sean $x=(x_1,x_2,x_3)$ y $y =(y_1,y_2,y_3)$, entonces
 \[
 x+y = (x_1 +y_1,x_2+ y_2, x_3 + y_3)= (y_1+x_1, y_2+x_2, y_3 + x_3)= y + x
 \]
\item[b)] Para todos $x,y,z \in \Bbb{R}^3$, se cumple que $(x + y) + z =x + (y + z)$ (asociatividad).
Sean $x=(x_1,x_2,x_3),$  $y=(y_1,y_2,y_3),$ y $z=(z_1,z_2,z_3)$, entonces
\begin{align*}
(x + y) + z &=((x_1,x_2,x_3) + (y_1,y_2,y_3))+ (z_1,z_2,z_3) \\
&=(x_1 +y_1,x_2+y_2,x_3 + y_3) +(z_1,z_2,z_3) \\ &=((x_1 +y_1) +
z_1,(x_2+y_2) + z_2,(x_3 + y_3) + z_3)  \\ & = (x_1 + (y_1 +
z_1),x_2+ (y_2 + z_2),x_3 + (y_3 + z_3)) \\ &= (x_1,x_2,x_3) +
((y_1,y_2,y_3)+ (z_1,z_2,z_3)) \\ &= x + (y + z)
\end{align*}
\item[c)] Existe un elemento en $\Bbb{R}^3$ llamado cero y denotado por  0 tal que $0 + x = x$, pata todo $x \in \Bbb{R}^3$ (existencia del neutro aditivo).
Sea $0 =(0,0,0)$ entonces si $x = (x_1,x_2,x_3)$ tenemos
\[0 + x =(0,0,0) + (x_1,x_2,x_3) = (0 + x_1, 0 + x_2, 0 + x_3) =
(x_1,x_2,x_3) =x
\]
\item[d)] Para todo $x \in \Bbb{R}^3$ existe un elemento $-x$ tal que $x + (-x) = 0$ (existencia de elementos inversos).
Sea $x \in \Bbb{R}^3$, con $x =(x_1,x_2,x_3)$, definimos el inverso
de $x$ por $-x=(-x_1,-x_2,-x_3)$, entonces tenemos
\begin{align*}
 x + (-x) &= (x_1,x_2,x_3) + (-x_1,-x_2,-x_3)\\
  & = (x_1 + (-x_1), x_2 + (-x_2), x_3 +
 (-x_3)) \\   &= (x_1 -x_1, x_2 -x_2, x_3-x_3) = (0,0,0) = 0
\end{align*}
\end{enumerate}
Ahora veamos que $\Bbb{R}^3$ con la operaci\'on producto $\cdot$
cumple
\begin{enumerate}
 \item[a)] Para todo $x \in \Bbb{R}^3$, se tiene que $1 \, x = x$, con $1 \in \Bbb{R}.$
Si $x \in \Bbb{R}^3$,
\[
 1 \cdot x = 1 \cdot (x_1,x_2,x_3) = (1 \cdot x_1, 1 \cdot x_2, 1 \cdot
 x_3) = (x_1,x_2,x_3) = x
\]

 \item[b)] Para todo $x \in \Bbb{R}^3$ y para todo $\lambda$ y $\mu \in \Bbb{R}$, se tiene que $\lambda(\mu x) = (\lambda\mu) x$.
 Sea $x \in \Bbb{R}^3$, con $x = (x_1,x_2,x_3)$, tenemos
 \begin{align*}
 \lambda(\mu x) &=\lambda(\mu (x_1,x_2,x_3)) = \lambda (\mu x_1,\mu
x_2,\mu x_3)   = (\lambda(\mu x_1),\lambda (\mu x_2),\lambda (\mu
x_3)) \\ & = ((\lambda\mu) x_1,(\lambda \mu) x_2,(\lambda \mu) x_3)
 = (\lambda\mu) x
 \end{align*}
 \item[c)] El producto por escalar es distributivo, es decir,
 \begin{align*}
(\lambda + \mu) x &=\lambda x + \mu x, \\
\lambda(x + y) &= \lambda x + \lambda y,
 \end{align*}
para todos $\lambda,\mu \in K$ y para todos $x, y \in V$.

\vspace{.2cm}

Sea $x \in \Bbb{R}^3$, con $x = (x_1,x_2,x_3)$, tenemos
\begin{align*}
 (\lambda + \mu) \cdot x & = (\lambda +\mu)\cdot (x_1,x_2,x_3) = ((\lambda +\mu) x_1,
(\lambda +\mu) x_2,(\lambda +\mu) x_3)  \\
&= (\lambda x_1 + \mu
x_1,\lambda x_2 + \mu x_2,\lambda  x_3 + \mu x_3) \\
& = (\lambda
x_1,\lambda x_2,\lambda x_3) + (\mu x_1,\mu x_2,\mu x_3) =
 \\ &= \lambda x + \mu x.
 \end{align*}
\begin{align*}
 \lambda  \cdot ( x + y) &= \lambda \cdot ((x_1,x_2,x_3)+ (y_1,y_2,y_3)) = \lambda \cdot
 (x_1 + y_1, x_2 + y_2, x_3 + y_3)  \\ &= (\lambda (x_1 + y_1),\lambda (x_2 +  y_2),\lambda (x_3 + y_3)) \\
 & = (\lambda x_1,\lambda x_2,\lambda x_3) + (\lambda
y_1,\lambda y_2,\lambda y_3) =
 \\ &= \lambda x + \lambda y.
 \end{align*}

para todos $\lambda,\mu \in K$ y para todos $x, y \in V$
\end{enumerate}
\end{ejemplo}


\begin{definicion}
Sea $W$ un subconjunto no vac\'{\i}o de $V,$ se dice que $W$
es un\textbf{ subespacio vectorial} de $V,$ si satisface las
siguientes propiedades:

\begin{enumerate}
 \item Para todos $x$ y $y \in W,$ se tiene que $x+y \in W,$ es decir, $W$ es cerrado bajo la suma.
 \item Para todo $x \in W$ y para todo $\lambda \in \Bbb{R}$, $\lambda
  x \in W,$ es decir $W$ es cerrado bajo producto por escalar.
\end{enumerate}
\end{definicion}

\begin{ejemplo}
Sea
\[
  W = \{(x_1,x_2,x_3) \in \Bbb{R}^3 \mid x_3 = 0  \}
\]
es decir, $x \in W$, entonces $x =(x_1,x_2, 0)$. Entonces $W$ es un
subespacio vectorial de $\Bbb{R}^3.$ Para esto verifiquemos que si
$x, y \in W,$ entonces $x + y \in W$. Como $x,y \in W,$ $x=(x_1,x_2, 0)$ y $y =(y_1,y_2, 0 )$, luego $x + y = (x_1+y_1,x_2 +
y_2, 0) \in W $. Ahora veamos que si $x \in W$ y $\lambda \in
\Bbb{R}$, $\lambda x \in W$, lo cual se sigue de que si $x =
(x_1,x_2, 0)$, entonces $\lambda x = \lambda(x_1,x_2, 0) =
(\lambda x_1,\lambda x_2, 0) \in W$.
\end{ejemplo}

\begin{ejemplo}
Sea $A$ una matriz 3 por 2. Entonces
\begin{enumerate}
\item[a)] el espacio columna de $A$, el cual es el conjunto de todas las combinaciones lineales de las
columnas de $A$ y se le denota por $C(A)$ es un subespacio de
$\Bbb{R}^3$
\item[b)] el espacio nulo de $A$, que consta de todos los vectores $x$ tales que $Ax= 0$
y  se le denota por $N(A)$ es un subespacio de $\Bbb{R}^2$
\item[c)] el espacio rengl\'on de $A$, generado por los renglones de
$A$, el cual es el espacio columna de $A^T$ y se le denota por
$C(A^T)$ es un subespacio de $\Bbb{R}^2$
\item[d)] el espacio nulo izquierdo de $A$ el cual es espacio nulo
de $A^T$,  denotado por $N(A^T)$, es un subespacio de $\Bbb{R}^3$.
\end{enumerate}
\end{ejemplo}


\newpage
\textbf{TAREA: ESPACIOS Y SUBESPACIOS VECTORIALES}

\vspace{.2cm}


\begin{enumerate}
\item Demostrar que el conjunto $V$ de matrices $3 \times 3$, es un espacio
vectorial sobre  $\Bbb{R}$ con las operaciones suma  y producto por
escalar usuales, es decir:

Si $A = [a_{ij}], B = [b_{ij}]$ matrices 3 por 3. La operaci�n
\textbf{ suma} de $A$ con $B$ es:


\begin{equation*}
 \left[
  \begin{matrix}


     a_{11}   & a_{12}   & a_{13} \\
     a_{21}   & a_{22}  & a_{23} \\
      a_{31}  & a_{32}  & a_{33}
  \end{matrix}
  \right] +
 \left[
  \begin{matrix}
     b_{11}   & b_{12}   & b_{13} \\
     b_{21}   & b_{22}  & b_{23} \\
      b_{31}  & b_{32}  & b_{33}
  \end{matrix}
  \right] =
  \left[
  \begin{matrix}
     a_{11} + b_{11}  & a_{12} + b_{12}  & a_{13} + b_{13} \\
     a_{21} + b_{21}   & a_{22} + b_{22}  & a_{23} + b_{23} \\
      a_{31} + b_{31}  & a_{32} + b_{32}  & a_{33} + b_{33}
  \end{matrix}
\right]
\end{equation*}

y producto de una matriz por un escalar:
\[ \lambda  \cdot A = \lambda
 \left[
  \begin{matrix}
     a_{11}   & a_{12}   & a_{13} \\
     a_{21}   & a_{22}  & a_{23} \\
      a_{31}  & a_{32}  & a_{33}
  \end{matrix}
  \right]
  =
\left[
  \begin{matrix}
    \lambda a_{11}   &\lambda  a_{12}   & \lambda a_{13} \\
     \lambda a_{21}   &\lambda a_{22}  & \lambda a_{23} \\
    \lambda  a_{31}  &\lambda a_{32}  &\lambda a_{33}
  \end{matrix}
  \right]
\]

\item Una matriz (cuadrada) 3 \times 3 \  $\left[a_{ij} \right]$ sobre $\Bbb{R}$  es sim\'etrica si
$a_{ij} = a_{ji}$ para todo $i,j$. Demostrar que las matrices sim\'etricas forman un subespacio del espacio de las
matrices $3 \times 3$.

\item Sea $V$ el conjunto de todas las matrices $2 \times 2$ sobre $\Bbb{R}$. Demostrar que $V$ es un espacio vectorial sobre $\Bbb{R}$ con las operaciones usuales de suma y producto por escalar usuales. Sea $W$ el subconjunto de $V$ que consta de las matrices de la forma
\[
\left[
  \begin{matrix}
   a_1    & a_2 \\
   a_3  &  0
  \end{matrix}
 \right]
 \]
con $a_1,a_2,a_3 \in \Bbb{R}$. Demostrar que $W$  es un subespacio
vectorial de $V$.
\item  Demostrar que los siguientes conjuntos de vectores $\alpha = (x_1,x_2,x_3) \in \Bbb{R}^3$ son subespacios vectoriales de $\Bbb{R}^3$
\begin{enumerate}
\item Todos los $\alpha$, tales que $x_1 \geq 0$.
\item Todos los $\alpha$, tales que $x_1 + 3x_2 = x_3$.
\end{enumerate}
\item

\end{enumerate}



\newpage

\subsection{Combinacion lineal de vectores, dependencia e independencia lineal}

\begin{definicion}
Un vector $\beta \in V$, se dice \textbf{combinaci\'on lineal} de los vectores $\alpha_1,...,\alpha_n \in
 V$,  si existen escalares $a_1,...,a_n \in K$, tales que:
\[
  \beta = \sum_{i=1}^{n} a_i \alpha_i.
\]
\end{definicion}

\begin{ejemplo}
El vector $ \left(
\begin{matrix}
6  \\
7
\end{matrix}
\right)$ en $\Bbb{R}^2$ es combinaci\'on lineal de los vectores
\[
\left(
\begin{matrix}
1  \\
0
\end{matrix}
\right) \quad \text{y} \quad \left(
\begin{matrix}
0  \\
1
\end{matrix}
\right)
\]
ya que:

\[ 6 \cdot
\left(
\begin{matrix}
1  \\
0
\end{matrix}
\right) + 7 \cdot  \left(
\begin{matrix}
0  \\
1
\end{matrix}
\right) = \left(
\begin{matrix}
6  \\
7
\end{matrix}
\right)
\]
\end{ejemplo}

\begin{definicion}
Sea $S$ es cualquier colecci\'on de vectores de $V$. \textbf{El sub\-es\-pa\-cio generado} por $S$ se define como
\[
  L(S) = \{ \sum_{i=1}^{k} a_i \alpha_i \mid a_i \in K, \alpha_i \in S \ \text{y}   \  k=1,2,3,... \}
\]
\end{definicion}

Cuando $L(S) = V$, decimos que $S$ genera a $V$


\begin{definicion}
Un subconjunto $S$ de $V$ se dice $\textbf{linealmente dependiente}$, si existen vectores distintos
$\alpha_1,...,\alpha_n$ de $S$ y escalares $a_1,...,a_n \in K$, no todos cero, tales que:
\[
     a_1 \alpha_1 + \cdots + a_n \alpha_n = 0.
\]
Un conjunto que no es linealmente dependiente se dice \textbf{linealmente independiente}. Si el conjunto $S$ solo tiene un n\'umero finito de vectores $\alpha_1,...,\alpha_n$, se dice a veces que los $\alpha_1,...,\alpha_n$ son dependientes (o independientes), en vez de decir que $S$ es dependiente (o independiente).
\end{definicion}

\begin{ejemplo}
Los siguientes vectores en $\Bbb{R}^2$
\[
\left(
\begin{matrix}
2  \\
3
\end{matrix}
\right) \quad \text{y} \quad \left(
\begin{matrix}
-1  \\
5
\end{matrix}
\right)
\]
son linealmente independientes.

\vspace{.2cm}

\textbf{Soluci\'on}: Sea
\[ a_1 \cdot
\left(
\begin{matrix}
2  \\
3
\end{matrix}
\right)  +  a_2 \cdot \left(
\begin{matrix}
-1  \\
5
\end{matrix}
\right)
=
\left(
\begin{matrix}
0  \\
0
\end{matrix}
\right)

de lo anterior obtenemos el sistema de ecuaciones:
\begin{align*}
 2a_1 - a_2 &= 0 \\
 3a_1 + 5a_2 &= 0
\end{align*}
\]
el cual tiene como soluci\'on: $a_1  = 0$ y $a_2 = 0$.
\end{ejemplo}


\begin{ejemplo}
Los vectores

\[
\left(
\begin{matrix}
2  \\
3
\end{matrix}
\right) \quad \text{y} \quad \left(
\begin{matrix}
4  \\
6
\end{matrix}
\right)
\]
son linealmente dependientes. Esto se sigue de
\[ -2 \cdot
\left(
\begin{matrix}
2  \\
3
\end{matrix}
\right) + \left(
\begin{matrix}
4  \\
6
\end{matrix}
\right) = \left(
\begin{matrix}
0  \\
0
\end{matrix}
\right)
\]
\end{ejemplo}

\begin{ejemplo}
Los vectores $(1,2,3)$ y $(1,1,0)$ son linealmente independientes en
$\Bbb{R}^3$.

\vspace{.2cm} Sea:
\[
a_1 \cdot (1,2,3) + a_2 \cdot  (1,1,0) = ( 0,0,0)
\]

Entonces
\begin{align*}
 a_1  + a_2  &= 0 \\
 2a_1 + a_2  &= 0 \\
 3a_1 &= 0
\end{align*}
Es f\'acil ver que el sistema de ecuaciones anterior tiene como
\'unica soluci\'on $a_1 = a_2 =  0$.
\end{ejemplo}

\begin{ejemplo} Demostrar
\medskip
\begin{enumerate}
\item[(a)] S�
$\alpha_1 = \left( \begin{matrix} 3 \\  1 \end{matrix}
\right)$ y $\alpha_2 =  \left(\begin{matrix} 6 \\  2 \end{matrix}
\right) $ son linealmente dependientes.
\item[(b)] S�
$\alpha_1 = \left( \begin{matrix} 3 \\  1 \end{matrix}
\right)$ y $\alpha_2 =  \left(\begin{matrix} 1 \\  2 \end{matrix}
\right) $ son linealmente independientes.
\end{enumerate}
\end{ejemplo}

\vspace{.2cm}

\textbf{Soluci\'on:}

\vspace{.2cm}

\begin{enumerate}
\item[(a)] Se ve que $\alpha_2 =2 \alpha_1$, luego $2\alpha_1 -
  \alpha_2 =0$. Tomando $a_1 = 2$ y $a_2 = -1$ se obtiene $a_1
  \alpha_1 + a_2 \alpha_2 = 0$, lo cual prueba que $a_1$ y $a_2$ son
  linealmente dependientes.
\item[(b)] La ecuaci\'on $a_1 \alpha_1 + a_2 \alpha_2 = 0$, da lugar
  al sistema
\begin{align*}
3a_1 + a_2 & = 0 \\
 a_1 + 2a_2 &= 0
\end{align*}
que tiene como soluci\'on \'unica $a_1 = a_2 = 0$. Por tanto
$\alpha_1$ y $\alpha_2$ son linealmente independientes.
\end{enumerate}


\subsection{Base y dimensi\'on}
\begin{definicion}
Una \textbf{base} de $\Bbb{R}^n$ es un conjunto de vectores linealmente independientes de $\Bbb{R}^n$ y que genera el
espacio $\Bbb{R}^n$.
\end{definicion}

\begin{teorema}
Sea $\{\alpha_1,...,\alpha_n\}$ un subconjunto de $\Bbb{R}^n$, entonces las siguientes condiciones son equivalentes:
\begin{enumerate}
  \item El conjunto $\{\alpha_1,...,\alpha_n\}$ es una base.
  \item El conjunto $\{\alpha_1,...,\alpha_n\}$ es linealmente independiente.
  \item El conjunto $\{\alpha_1,...,\alpha_n\}$ genera a $\Bbb{R}^n$.
\end{enumerate}
\end{teorema}


\begin{ejemplo}

Los vectores

\[
\left(
\begin{matrix}
1 \\
1
\end{matrix}
\right) \quad \text{y} \quad \left(
\begin{matrix}
-1  \\
1
\end{matrix}
\right)
\]
son una base de $\Bbb{R}^2$. Si


\[ a_1 \cdot
\left(
\begin{matrix}
1 \\
1
\end{matrix}
\right) +  a_2 \cdot \left(
\begin{matrix}
-1  \\
1
\end{matrix}
\right)
=
\left(
\begin{matrix}
0 \\
0
\end{matrix}
\right)
\]
Entonces de la combinaci\'on lineal anterior, obtenemos el sistema
de ecuaciones:
\begin{align*}
a_1  - a_2 &= 0 \\
a_1 + a_2 &= 0
\end{align*}
El cual tiene como \'unica soluci\'on: $a_1 = a_2  = 0$.  Ahora, sea
$(x,y) \in \Bbb{R}^2$, veamos que existen $a_1,a_2 \in \Bbb{R}$
tales que $a_1 \cdot (1,1) + a_2 \cdot (-1,1) = (x,y)$. Es f\'acil
ver que $a_1 = \frac{x + y}{2}$ y $a_2 = \frac{y - x}{2}$. De lo
anterior se sigue que los vectores $(1, 1)$ y $(-1, 1)$ son
linealmente independientes y que generan a $\Bbb{R}^2$, por lo tanto
son una base de $\Bbb{R}^2$.
\end{ejemplo}

\begin{definicion}
Dos bases cualesquiera de un espacio vectorial $V$  contiene el
mismo n\'umero de vectores. Este n\'umero que es compartido por
todas las bases y expresa el n\'umero de grados de libertad del
espacio, es la dimensi\'on de $V$.
\end{definicion}

\begin{ejemplo}
En $\Bbb{R}^n$, sean $e_i = (0,...,0,1,0,...,0)$ donde el 1 aparece
en el $i$-\'esimo lugar y todas las otras coordenadas son cero. El
conjunto $\{e_i\}_{i = 1}^{n}$ es una base de $\Bbb{R}^n$ llamada la
\textbf{base can\'onica}, por lo tanto la dimensi\'on del espacio
$\Bbb{R}^n $ es $n$.
\end{ejemplo}
\begin{ejemplo}
Si $A$ es una matriz  3 por 2 con rango $r$, entonces:
\begin{enumerate}
\item[a)] La dimensi\'on del espacio columna $C(A)$ es el rango $r$.
\item[b)] La dimensi\'on del espacio nulo de $A$ es $2-r$.
\item[c)] La dimensi\'on del espacio rengl\'on $C(A^T)$ es tambi\'en
$r$.
\item[d)] La dimensi\'on del espacio nulo izquierdo $N(A^T) =3- r.$
\end{enumerate}
\end{ejemplo}


\newpage
\textbf{TAREA: BASES DE ESPACIOS VECTORIALES}

\vspace{.2cm}

\begin{enumerate}
\item Decida la dependencia o independencia en $\Bbb{R}^2$ y  $\Bbb{R}^3$ de
\begin{enumerate}
\item[a)] los vectores
    $(1,1)$ y $(1,-2)$.
\item[b)] los vectores $(1,-3,2),(2,1,-3)$ y $(-3,2,1)$.
\item[c)] los vectores
    $(1,3,2),(2,1,3)$ y $(3,2,1)$.
\end{enumerate}

\item Demuestre que el siguiente subconjunto de las matrices $2 \times
2$
\[
\left\{
 \left(
\begin{matrix}
0 & 1 \\
0 &1
\end{matrix}
\right),
\left(
\begin{matrix}
0 & 1 \\
0 & 1
\end{matrix}
\right) \right\}
\]

es linealmente independiente

\item Demostrar que los siguientes vectores forman una base para $\Bbb{R}^2$.
\begin{enumerate}
\item[a)]
\[
\alpha_1=(1,1), \quad \alpha_2=(1,-1)
\]
\item[b)]
\[
\alpha_1=(-1,1), \quad \alpha_2=(-1,0)
\]
\end{enumerate}
\item Demostrar que los siguientes vectores forman una base para $\Bbb{R}^3$.
\item[a)]
\[
\alpha_1=(1,0,1), \quad \alpha_2=(1,1,0), \quad \alpha_3=(0,1,1)
\]
\item[b)]
\[
\alpha_1=(1,0,-1), \quad \alpha_2=(1,2,1), \quad \alpha_3=(0,3,-2)
\]
\item[c)]
\[
\alpha_1=(1,0,-1), \quad \alpha_2=(1,1,1), \quad \alpha_3=(1,0,0)
\]

\item Encuentre una base para cada uno de los siguientes subespacios
de $\Bbb{R}^4$
\begin{enumerate}
\item[a)] Todos los vectores cuyas componentes son iguales.
\item[b)] Todos los vectores tales que la suma de sus componentes es
cero.
\end{enumerate}
\item Encuentre una base para cada uno de los siguientes subespacios
de matrices 3 por 3:
\begin{enumerate}
\item[a)] Todas las matrices diagonales
\item[b)] Todas las matrices sim\'etrica
\item[c)] Todas las matrices sesgadas sim\'etricas (A^T = -A)
\end{enumerate}
\item Encontrar la dimensi�n y una base para la siguiente matriz.\\
\[
\left(
\begin{matrix}
2& 1& 2\\
0& 3& -1\\
4& 1& 1
\end{matrix}
\right)
\]
\item Encontrar la dimensi�n y una base para la siguiente matriz.\\
\[
\left(
\begin{matrix}
1& 1& 0& 0& -1\\
2& 2& 0& 1& 0\\
-1& 0& -1& 4& 2
\end{matrix}
\right)
\]
\item Sea $V$ el espacio vectorial de las matrices $2 \times 2$
sobre el campo $\Bbb{R}$. Demuestre que $V$ tiene dimensi\'on 4
encontrando una base de $V$ que tenga cuatro elementos.
\end{enumerate}

\newpage

\subsection{Transformaci\'on lineal}

\begin{definicion}
Sean $V$ y $W$ dos espacios vectoriales. Una transformaci\'on lineal
es una funci\'on  $T: V \to W$ que satisface las siguientes
propiedades:
\begin{enumerate}
\item $ T(\alpha + \beta) = T(\alpha) + T (\beta)$, para todos
$\alpha, \beta \in V.$
\item $T(r \alpha) = r T(\alpha)$, para todo escalar $r \in \Bbb{R}$
 y para todo  $\alpha \in V .$
\end{enumerate}
Un operador lineal sobre $V$ es una transformaci\'on lineal de $V$
en si mismo.
\end{definicion}
\begin{ejemplo}
La funci\'on $0 : V \to W$ definida por $0(v) = 0$ que mapea todos
los elementos del espacio vectorial $V$ al elemento cero del espacio
$W,$ es claramente una funci\'on lineal, llamada la transformaci\'on
cero.
\end{ejemplo}
\begin{ejemplo}
La funci\'on $1V : V \to V$ dada por $1V (v) = v$ es un operador
lineal denominado operador identidad sobre $V$.
\end{ejemplo}



\begin{ejemplo}
Si $V = \Bbb{R}^n$ y $W  = \Bbb{R}^m$ las trasformaciones lineales
entre $V$ y $W$ corresponden a las matrices $A$ de $m \times n.$ En
particular, si
\[
A =
\left(
\begin{matrix}
1 & 1 & -1 \\
1 & -1 & 1
\end{matrix}
\right)
 entonces $T_A : \Bbb{R}^3  \to \Bbb{R}^2$ dada por $T_A (x) = Ax$ es
 lineal, dada por:
\[
 T_A
 \left(
\begin{matrix}
x \\
y \\
z
\end{matrix}
\right) =
\left(
\begin{matrix}
x+y-z \\
x-y+z
\end{matrix}
\right)
\end{ejemplo}



\begin{teorema}
Si $T : V \to W$ es una tranformaci\'on lineal, y si
$\alpha_1,...,\alpha_n$  son vectores de $V,$ entonces dados los
escalares $a_1,a_2,...,a_n,$
\[
   T(a_1 \alpha_1 + \cdots + a_n \alpha_n) = a_1 T(\alpha_1) + \cdots
   + a_n T(\alpha_n).
\]
\end{teorema}

\begin{teorema}
Sea $T : V \to W$ una tranformaci\'on lineal, si $\{\alpha_1,...,
\alpha_n \}$ es una base para $V,$  y si $T(\alpha_i) = \beta_i$, $i=
1, 2,...n$ entonces para cualquier vector $\alpha \in V,$ $T(\alpha)$
est\'a determinada y $T(\alpha) = a_1 \beta_1 + \cdots +  a_n
\beta_n$, donde $a_1, a_2,...,a_n$ son escalares tales que $\alpha =
a_1 \alpha_1 + \cdots + a_n \alpha_n$.
\end{teorema}

\begin{teorema}
\label{T}
  Sea $V$ un espacio vectorial de dimensi\'on finita $n$, sea
  $\alpha_1,...,\alpha_n$ una base ordenada de $V$. Sean $W$  un
  espacio vectorial y $\beta_1,...,\beta_n$ vectores cualesquiera en
  $W.$ Entonces existe una \'unica tranformaci\'on lineal $T : V \to
  W$ tal que
\[
  T(\alpha_j) = \beta_j. \quad  j=1,...,n.
\]
\end{teorema}

\begin{ejemplo}
Consideremos la base de $\Bbb{R}^2$ formada por lo vectores $\alpha_1 = (1,2), \alpha_2= (3,4)$. Por el Teorema \ref{T}
existe una transformaci\'on lineal $T$ de $\Bbb{R}^2$  en $\Bbb{R}^2$ tal que
\begin{align*}
T(\alpha_1) &= (3,2,1) \\
T(\alpha_2) &= (6,5,4)
\end{align*}
Encontremos $T(1,0)$.  Si $(1,0) = c_1(1,2) + c_2(3,4)$, con $c_1,c_2 \in \Bbb{R}$, entonces $c_1= -2$ y $c_2= 1$. Por lo tanto
\begin{align*}
T((1,0) ) &= T(c_1(1,2) + c_2(3,4)) \\ &= c_1 T(1,2) + c_2 T(3,4) \\ & =  -2(3,2,1) + (6,5,4) \\ & = (0,1,2).
\end{align*}
\end{ejemplo}


\subsection{Nucleo e imagen}

\begin{definicion}
Sea   $T : V  \to W$ una tranformaci\'on lineal. Definimos el
n\'ucleo de $T$ como el conjunto $N_T = \{\, \alpha \in V \mid T
(\alpha) = 0 \}$. La imagen de $T$, denotada $R_T$, se define como
$R_T = \{\beta \in W   \mid \text{existe un} \,  \alpha \in V
\text{y
  satisface} \,  T(\alpha) = \beta \}$.
\end{definicion}

\begin{ejemplo}
Sea $T: \Bbb{R}^3 \to \Bbb{R}^2$ dada por $T(x,y,z) = (-x + 3y + z, y + 2z)$. Entonces
$(x,y,z) \in N_T$ si y solo si
\begin{align*}
 -x + 3y + z &= 0 \\
 y + 2z &= 0
\end{align*}
La forma escalonada reducida de la matriz de los coeficientes de este sistema es
\[
\left(
\begin{matrix}
1 & 0 & 5 \\
0 & 1 & 2
\end{matrix}
\right).
\]
Por tanto el n\'ucleo de $T$, $N_T = \{(-5z,-2z, z) | z \in \Bbb{R}\}$.
\end{ejemplo}

El siguiente teorema  es uno de los m\'as importantes en la teor\'ia
de espacios vectoriales  de dimensi\'on finita.

\begin{teorema}
Sean $V$ y $W$ espacios vectoriales de dimensi\'on finita,
$T : V \to W$ una tranformaci\'on lineal, entonces la siguiente
ecuaci\'on se cumple:
\[
   dim(V) =dim (N_T) + dim(R_T)
\]
\end{teorema}

\subsection{Matriz de una tranformaci\'on lineal}
Sabemos que una tranformaci\'on lineal  queda completamente
determinada en una base. Si $T: V \to W$  es una tranformaci\'on
lineal, $\{ \alpha_1,..., \alpha_n\}$ y $\{ \beta_1,...,\beta_m \}$
son bases de $V$ y $W$ respectivamente, entonces para cada $j =
1,...,n$, $T(\alpha_j)$ se representa como combinaci\'on linel de
los elementos de la base $ \{ \beta_1,...,\beta_m \}$, es decir.
existen escalares $a_{1j},...,a_{mj}$, \'unicos, tales que:
\[
    T(\alpha_j) = \sum_{i = 1}^{m} a_{ij} \beta_i
\]

Los escalares $a_{ij}$ solamente dependen de la tranformaci\'on
lineal y de las bases elegidas, con ellos formamos la matriz:

\begin{equation*}
A  =
 \left[
  \begin{matrix}
     a_{11}   & a_{12}  & \cdots   & a_{1n} \\
     a_{21}   & a_{22}  & \cdots   & a_{2n} \\
    \vdots    & \vdots  & \cdots   & \vdots \\
      a_{m1}  & a_{m2}  & \cdots  & a_{mn}
  \end{matrix}
  \right]
\end{equation*}


\begin{ejemplo}
Sea $T: \Bbb{R}^2 \to \Bbb{R}^2$ dada por $T(x,y) = (x,0)$. Entonces la matriz asociada a
$T$ respecto de las bases can\'onicas es
\[ A =
\left[
  \begin{matrix}
   1    & 0     \\
   0  &  0
  \end{matrix}
  \right]
\]
\end{ejemplo}



\begin{ejemplo}
Sea $T: \Bbb{R}^2 \to \Bbb{R}^2$ dada por $T(x,y) = (2x + y, x - y)$. Entonces la matriz asociada a
$T$ respecto de las bases can\'onicas es
\[ A =
\left[
  \begin{matrix}
   2    & 1     \\
   1  &  -1
  \end{matrix}
  \right]
\]
\end{ejemplo}


\subsection{Cambio de base}

\begin{teorema}
Sean $V$ un espacio vectorial de dimensi\'on $n$ sobre el cuerpo $F$, y $W$ un espacio vectorial de dimensi\'on $n$ sobre
$F$. Sean $B$ una base ordenada de $V$ y $B'$ una base ordenada de $W$. Para cada transformaci\'on lineal $T$ de $V$ en $W$, existe una matriz $m \times n$, $A$, cuyos elementos pertenecen a $F$, tal que
\[
 [T \alpha]_{B'} = A [\alpha]_{B}
\]
Para todo vector $\alpha \in V.$
 \end{teorema}




\begin{definicion}
La matriz $A$ se llama \textbf{  la matriz asociada a la
trans\-for\-ma\-ci\'on $T$ respecto a las bases $\{ \alpha_1,...,\alpha_n \}$
y  $\{\beta_1,...,\beta_n \}$}.
\end{definicion}


\begin{ejemplo}
Sea $B$ la base de $\Bbb{R}^2$ formada por lo vectores $\alpha_1 = (1,1)$ y $\alpha_2 = (3,-2)$. Por el Teorema \ref{T}, existe
una \'unica transformaci\'on  lineal $T: \Bbb{R}^2 \to \Bbb{R}^2$ tal que $T(\alpha_1) = (4,5)$ y $T(\alpha_2) =(6, -1)$. Encontremos la matriz $A$ asociada a $T$ respecto a la base can\'onica de $\Bbb{R}^2$. Para determinar $A$, debemos determinar
$T(e_1) =(a,b)$ y $T(e_2) = (c,d)$. De las ecuaciones
\begin{align*}
T(1,1) &= T(e_1) + T(e_2) \ \text{y} \\
T(3,-2) &=  3 T(e_1) - 2T(e_2)
\end{align*}
se sigue que
\begin{align*}
(4,5) &= (a + c, b + d) \\
(6,-1) &= (3a- 2c, 3b - 2d)
\end{align*}
de donde se tiene el siguiente sistema de ecuaciones:
\begin{align*}
a+ c &=  4 \\
b+d &= 5 \\
3a -2c &= 6 \\
3b - 2d &= -1
\end{align*}
de cuya soluci\'on se sigue que  $T(e_1) = \left(\frac{14}{5}, \frac{9}{5} \right)$ y
$T(e_2) = \left(\frac{6}{5}, \frac{16}{5} \right)$, por lo tanto la matriz asociada a $T$ respecto a la
base can\'onica es:

\[ A =
\left[
  \begin{matrix}
   \frac{14}{5}    & \frac{6}{5}     \\
   \frac{9}{5}  &  \frac{16}{5}
  \end{matrix}
  \right]
\]

Por otro lado, la expresi\'on que define a $T(x,y)$ se obtiene del siguiente producto de matrices:
\[
AX = \left[
  \begin{matrix}
   \frac{14}{5}    & \frac{6}{5}     \\
   \frac{9}{5}  &  \frac{16}{5}
  \end{matrix}
  \right]
  \left[
  \begin{matrix}
   x\\
   y
  \end{matrix}
  \right]
  =
  \left[
  \begin{matrix}
   \frac{14}{5}x + \frac{6}{5}y     \\
   \frac{9}{5} x +  \frac{16}{5}y
  \end{matrix}
  \right]
\]
Por lo tanto $T(x,y) = (\frac{14}{5}x + \frac{6}{5}y, \frac{9}{5} x +  \frac{16}{5}y )$.
\end{ejemplo}

\begin{teorema}
(Cambio de base). Sea $T: V \to W$ una transformaci\'on lineal. Supongamos que $A$ es la matriz asociada a $T$ respecto a bases dadas $\{\alpha_1,...,\alpha_n \}$ en $V$ y $\{\beta_1,...,\beta_n \}$  en $W$. Si las bases anteriores se cambian a nuevas bases $\{ \alpha_1', \alpha_2',...,\alpha_n' \}$ y $\{\beta_1',...,\beta_n' \}$, con matrices de cambio de base $P$ y $Q$ respectivamente y $B$ es la matriz asociada a $T$ en estas nuevas bases, entonces se tiene:
\[
 B = Q^{-1} A P.
\]
\end{teorema}
\begin{corolario}

Si $T: V \to V$ es una transformaci\'on lineal, $\alpha_i = \beta_i$ y $\alpha_i' = \beta_i'$ para todo $i = 1,...,n.$ Entonces la matriz asociada a $T$ respecto a la nueva base es $P^{-1}A P$, $P$ la matriz de cambio de base.
\end{corolario}

\begin{ejemplo}
Sea $T: \Bbb{R}^3 \to \Bbb{R}^3$ dada por $T(x+ y - z, 2x - y + 3z, x- z)$. Para encontrar la matriz asociada a $T$ respecto a la base $\{(1,2,0), (1,-1,0),(1,1,1) \}$, primero encontramos la matriz asociada a $T$ respecto a la base can\'onica, la cual se obtiene evaluando a $T$ en los vectores can\'onicos. Tenemos que $T(1,0,0) =(1,2,1)$, $T(0,1,0) = (1,-1-0)$ y $T(0,0,1)=(-1,3,-1)$, por lo que la matriz asociada a $T$ respecto de la base can\'onica es:

\[
A=
\left(
\begin{matrix}
1 & 1 & -1 \\
2 &-1 & 3 \\
1 & 0 &-1
\end{matrix}
\right)
\]

La matriz de cambio de base es
\[
P=
\left(
\begin{matrix}
1 & 1 & 1 \\
2 &-1 & 1 \\
0 & 0 & 1
\end{matrix}
\right)
\]
con
\[
P^{-1}=\frac{1}{3}
\left(
\begin{matrix}
1 & 1 & -2 \\
2 &-1 & -1 \\
0 & 0 & 3
\end{matrix}
\right)
\]
Aplicando el Teorema anterior obtenemos que la matriz asociada a $T$ respecto de la base $\{(1,2,0), (1,-1,0),(1,1,1) \}$
es:
\[
B=\frac{1}{3}
\left(
\begin{matrix}
1 & 1 & -2 \\
2 &-1 & -1 \\
0 & 0 & 3
\end{matrix}
\right)
\left(
\begin{matrix}
1 & 1 & -1 \\
2 &-1 & 3 \\
1 & 0 &-1
\end{matrix}
\right)
\left(
\begin{matrix}
1 & 1 & 1 \\
2 &-1 & 1 \\
0 & 0 & 1
\end{matrix}
\right)
=
\frac{1}{3}
\left(
\begin{matrix}
1 & 1 & 5 \\
5 &-4 & -2 \\
3 & 3 & 0
\end{matrix}
\right).
\]

\end{ejemplo}




\newpage
\textbf{TAREA: TRANSFORMACIONES LINEALES}

\vspace{.2cm}

\begin{enumerate}
\item ?`Cu\'ales de las siguientes funciones $T$ de $\Bbb{R}^2$ en $\Bbb{R}^2$ son tranformaciones lineales?
\begin{enumerate}
\item[(a)] $T(x,y)  = (1 + x, y)$
\item[(b)] $T(x,y)  = (y, x)$
\item[(c)] $T(x,y) = (x^2, y)$
\item[(d)] $T(x,y) =(x- y,0)$
\end{enumerate}
\item ?` Existe una tranformaci\'on lineal $T$ de $\Bbb{R}^3$ en $\Bbb{R}^2$ tal que $T(1,-1,1) =(1, 0)$ y $T(1,1,1) =(0,1)$?
\item Para cada una de las siguientes transformaciones lineales, encuentre su n\'ucleo y rango
\begin{enumerate}
\item[a)] Sea $T : \Bbb{R}^2 \to \Bbb{R}^2$ definida por  $T(x,y) = (x-y, 3x + 2y )$
\item[b)] Sea $T: \Bbb{R}^2 \to \Bbb{R}$ definida por $T(x,y) = x +  y$
\item[c)]  Sea $T_A : V \to V$, dada por $T_A(X) = AX$,  con $V$ el espacio vectorial de las matrices $2 \times 2$,
$ A =
\left[
  \begin{matrix}
   1    & 2     \\
   0  &  1
  \end{matrix}
  \right]
$
y
$ X =
\left[
  \begin{matrix}
   x    & y     \\
   z  &  w
  \end{matrix}
  \right]
  $
\end{enumerate}


\item Sea $T$ el operador lineal sobre $\Bbb{R}^2$ definido por
\[
 T(x,y) = (-y, x)
\]
\begin{enumerate}
\item[a)]?` Cu\'al es la matriz de $T$ en la base can\'onica de $\Bbb{R}^2.$
\item[b)] ?`Cu\'al es la matriz de $T$ respecto de la base ordenada en $\Bbb{R}^2$ formada por los vectores
$\alpha_1 =(1,2)$ y $\alpha_2 =(1,-1)$?
\end{enumerate}
\item Sea $T$ la transformaci\'on lineal de $\Bbb{R}^3$ en $\Bbb{R}^2$ definida por
\[
 T(x,y,z) = (x + y, 2z-x).
\]
Si $B$ es la base ordenada can\'onica de $\Bbb{R}^3$ y $B'$ es la base ordenada can\'onica de $\Bbb{R}^2$,
?` cu\'al es la matriz de $T$ respecto al par de bases $B, B'$.

\newpage


\item Sea $T$ el operador lineal en $\Bbb{R}^3$ definido por
\[
T(x,y,z) = (3x + z, - 2x + y, - x+ 2y + 4z)
\]
\begin{enumerate}
\item[a)]?` Cu\'al es la matriz de $T$ en la base can\'onica de $\Bbb{R}^3.$
\item[b)] ?`Cu\'al es la matriz de $T$ respecto de la base ordenada en $\Bbb{R}^3$ formada por los vectores
$\alpha_1 =(1,0,1)$, $\alpha_2 =(-1,2,1)$ y $\alpha_3 =(2,1,1)$?.
\end{enumerate}
\end{enumerate}

\newpage


\begin{thebibliography}{10}
   \bibitem{T1}
    Gilbert Strang, 2007. Algebra Lineal y sus aplicaciones,
    Thomson. 4a edici\'on.
   \bibitem{T2}
    Darell A. Turkington, 2007. Mathematical Tools for Economics,
    Blackwell Publishing.
   \bibitem{T3}
    Mike Rosser, 2003. Basic Mathematics for Economicsts,
    Routledge. Routledge. 2da. edici\'on.
   \bibitem{T4}
   Nakos George, 2004. Algebra Lineal con aplicaciones. Internacional
   Thomson Editores. 2a Ed, M\'exico.
\end{thebibliography}





\end{document}
